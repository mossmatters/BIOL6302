\documentclass[12pt]{article}
\usepackage{hyperref}
\usepackage{multicol}
\usepackage{lipsum}
\usepackage{soul}
\usepackage{hyphenat}
\usepackage{titlesec}
\usepackage{lscape}
\usepackage{geometry}
\titleformat{\paragraph}[hang]{\normalfont\normalsize\bfseries}{\theparagraph}{1em}{}
\titlespacing*{\paragraph}{0pt}{3.25ex plus 1ex minus .2ex}{0.5em}
\usepackage{graphicx}
\usepackage{listings}			%For formatted code blocks.
\usepackage[usenames]{xcolor}
\definecolor{medium-gray}{gray}{0.75}

\newcounter{terminal}
\lstnewenvironment{Terminal}[1][]{
	\renewcommand\lstlistingname{Terminal}
	\setcounter{lstlisting}{\value{terminal}}
	\lstset{
		basicstyle=\footnotesize\ttfamily,		%Small text with a monospaced font
		aboveskip=4pt,
		belowskip=4pt,
		language=bash,
		backgroundcolor=\color{medium-gray},
		showlines=true,
		showstringspaces=false,
		%literate={\$}{{{\$}}}1,		%Allow dollar signs to still be printed within code blocks.
		escapechar=!
		#1
%		escapeinside={(*@}{@*)}
		}
	}{\addtocounter{terminal}{1}}	


\textwidth=7in
\textheight=9.5in
\topmargin=-1in
\headheight=0in
\headsep=.5in
\hoffset  -.85in

\pagestyle{plain}
\pagenumbering{arabic}
\renewcommand{\thefootnote}{\fnsymbol{footnote}}
\frenchspacing

\title{Phylogenetics Project and Presentation\\ \large BIOL6304: Principles and Practice of Phylogenetics }

\begin{document}
\maketitle

\section{Learning Objectives}

After completing this activity, students will be able to:

\begin{itemize}
\item Conduct original research in phylogenetics.
\item Demonstrate understanding of tree thinking through an 8-10 minute presentation.
\item Provide constructive feedback to peers on their phylogenetics projects.
\item Write an original manuscript on phylogenetics in the style of a scientific publication.
\end{itemize}

\section{Introduction}

The purpose of the final paper is to give students experience conducting, evaluating, and writing original research in phylogenetics.
The paper should have the format and level of detail expected from a peer-reviewed publication.
Papers should be at least 10 pages long including figures, tables, and works cited.

You will record a 10 minute presentation and submit it on Blackboard for your peers to view.
Each presenter will be assigned three peer-reviewers to assess their presentation and provide feedback.
This feedback should be used to make improvements before submitting your final paper.


\section{Assignment Summary}

Presentations are due by the beginning of the scheduled class period on \textbf{Tuesday, December 1}.
Presentations may be recorded in Powerpoint or using other recording software (Zoom, OBS, etc).

Project papers will be due one week later on \textbf{Tuesday, December 12 by the end of the day (midnight)}, and may be submitted via Blackboard.
You should turn in your manuscript as a single file-- a Word or PDF document containing the text and figures, or a manuscript with figures separately as part of a single compressed ZIP archive.


\section{Presentation}

The purpose of the presentation is to practice sharing of research ideas and results to peers, a major aspect of academic research.
Prepare your presentation as if you were presenting at an academic conference. 
Your presentation should include background information and rationale for the research you conducted.
It should detail the methods used to address the research questions, and appropriate interpretation of results.
Your presentation should conclude with suggestions for improved or future research.

Your grade will also be based on appropriate constructive feedback provided to the other students. 
You will not be graded on the feedback from other students, but you should incorporate constructive criticisms when preparing your final paper. 
Instead, your grade in this portion (30\% of your presentation grade) will be based on the level of feedback you provide to the other students.
Each presenter will receive feedback from three peers plus the instructor.

\section{Grading Rubric}

10\% : Introduction, background, and clearly stated hypothesis \\
10\% : Explanation of methods\\
15\% : Organization of presentation material\\
15\% : Use of ``tree thinking'' language\\
20\% : Accurate interpretation of phylogenetic results\\
30\% : Appropriate constructive feedback on other students' presentations\\



\section{Final Project Paper}

This paper should be at least 10 pages long and have the format and level of detail expected from a peer-reviewed journal publication. 
It should be broken into the standard original research sections, explained below:

\paragraph{Introduction}
Introduce your project and the rationale behind the study. 
What is your central question?
Provide background information about your study organisms, discuss any hypotheses you are testing, and explain the data you've collected to address the questions.
Cite all sources appropriately.

\paragraph{Materials and Methods}
Fully explain your method of data collection and analysis as if they were to be published in a peer-reviewed paper.
Some considerations include:
\begin{itemize}
\item Where did your data come from? If it was published before, cite the source (paper or repository). If it was newly generated, describe how.
\item What phylogenetic methods did you employ? 
\item Were any parameter settings changed from the default? If so, what were they and why?
\item What statistical methods did you use to analyze the significance of your results?
\end{itemize}
A person knowledgeable about phylogenetics should be able to recreate your study from this description.
If using a command-line program, job submission script, or other computational tools, these may be submitted as supplemental information.


\paragraph{Results and Discussion}

Describe the results of your analysis both in the text and using figures and/or tables.
Your description of the results will be evaluated using the tree-thinking terminology we learned in class.
You should discuss all relevant results, and then address your central question.
Does the data you collected support your initial hypotheses?
How do your results compare to previous published analysis?
Finally, discuss the limitations of your study, and suggest improvements or promising areas for further research.

This section should include at least one figure, which should have a detailed caption and should be cited within the text.


\paragraph{Works Cited}
Background information and previously published research should be adequately cited.
Plagiarism, including the failure to appropriately cite sources, will not be tolerated.
Be sure to cite all programs used in your analysis.  
Phylogenetics and bioinformatics software usually has a manuscript associated with it.
Citations for R packages can usually be found using the \verb+citation()+ command.

Any citation format is allowed, as long as it is consistent.


\section{Grading Rubric}

15\% : Introduction, background, and clearly stated hypothesis \\
15\% : Explanation of methods and data sources\\
5\% : Clarity of writing\\
30\% : Use of ``tree thinking'' language\\
20\% : Accurate interpretation of phylogenetic results\\
10\% : Description of results in context of prior research\\
5\% : Works cited and in-text citations\\



\end{document}