\documentclass[12pt]{article}
\usepackage{hyperref}
\usepackage{multicol}
\usepackage{lipsum}
\usepackage{soul}
\usepackage{hyphenat}
\usepackage{titlesec}
\usepackage{lscape}
\usepackage{geometry}
\titleformat{\paragraph}[hang]{\normalfont\normalsize\bfseries}{\theparagraph}{1em}{}
\titlespacing*{\paragraph}{0pt}{3.25ex plus 1ex minus .2ex}{0.5em}
\usepackage{graphicx}
\usepackage{listings}			%For formatted code blocks.
\usepackage[usenames]{xcolor}
\definecolor{medium-gray}{gray}{0.75}

\newcounter{terminal}
\lstnewenvironment{Terminal}[1][]{
	\renewcommand\lstlistingname{Terminal}
	\setcounter{lstlisting}{\value{terminal}}
	\lstset{
		basicstyle=\footnotesize\ttfamily,		%Small text with a monospaced font
		aboveskip=4pt,
		belowskip=4pt,
		language=bash,
		backgroundcolor=\color{medium-gray},
		showlines=true,
		showstringspaces=false,
		%literate={\$}{{{\$}}}1,		%Allow dollar signs to still be printed within code blocks.
		escapechar=!
		#1
%		escapeinside={(*@}{@*)}
		}
	}{\addtocounter{terminal}{1}}	


\textwidth=7in
\textheight=9.5in
\topmargin=-1in
\headheight=0in
\headsep=.5in
\hoffset  -.85in

\pagestyle{plain}
\pagenumbering{arabic}
\renewcommand{\thefootnote}{\fnsymbol{footnote}}
\frenchspacing

\title{Tree Thinking Short Paper\\ \large BIOL6304: Principles and Practice of Phylogenetics \\ Caminalcules or Dendrogrammaceae Assignment }

\begin{document}
\maketitle

\section{Learning Objectives}

After completing this activity, students will be able to:

\begin{enumerate}
\item Construct a character matrix of morphological characters.
\item Infer a phylogeny using maximum parsimony.
\item Interpret phylogenies using the principles of tree thinking.
\item Write a systematics report based on a phylogenetic inference.
\end{enumerate}

\section{Introduction}

Over the last two weeks, you have been working with the Caminalcules or Dendrogrammaceae, and should have generated:
\begin{itemize}
\item A matrix of morphological characters.
\item A phylogeny inferred using maximum parsimony.
\item Reconstructed character states for one or more characters.
\end{itemize}

In this assignment, you are tasked with writing a systematics report of the Caminalcules or Dendrogrammaceae based on these data. 
The purpose of the assignment is for students to demonstrate an understanding of tree thinking and the 
You will also perform a phylogenetic comparison between two methods of inferring phylogenies.
Exactly what you decide to compare is up to you.

Note that if the phylogeny you inferred with your character matrix is completely unresolved, it will not be appropriate for this paper. 
You may add to or replace characters as needed, but a somewhat resolved phylogeny is required for the paper.

\section{What to turn in}

The full paper should be 3-5 pages long (not including figures/tables).
You should also turn in one or more NEXUS file(s) containing your character matrix and your phylogenetic trees.

Please turn in all of these files on Blackboard by \textbf{Midnight on September 29, 2022}.

\section{Sections of the Short Paper}

\paragraph{Introduction}
Introduce the Caminalcules or Dendrogrammaceae and the rationale behind the study. 
Perhaps you are interested in the evolution of a particular character, or you have chosen to compare two weighting schemes.
Explain in the text what your comparison might mean for the evolution of the Caminalcules or Dendrogrammaceae: what do you expect to find?

\paragraph{Materials and Methods}
Fully explain your choice of characters and character states for the Caminalcules or Dendrogrammaceae (a table may be appropriate).
In a few paragraphs, explain your procedure for inferring the phylogenetic tree of the Caminalcules or Dendrogrammaceae.
Which programs did you use? 
What parameter settings did you choose, and why?
How are you evaluating support for your phylogeny?
The goal with any methods section is for someone to take your NEXUS file and instructions and completely recreate your study.

\textbf{You must choose some kind of comparison to do within your paper}, which can focus on either phylogenetic methods or the character matrix itself.
Describe the comparison you are making and all of the instructions needed to repeat the comparison.
The goal of the Comparison portion is to test how you evaluate similar trees using tree thinking terminology.


\paragraph{Results and Discussion}
Present the results of the phylogenetic inference as both a figure and in the text.
Using the tree terminology and tree-thinking methods we have discussed in class, describe the evolution of the Caminalcules or Dendrogrammaceae.
How are the groups related, and what does that mean for character evolution?
Present the results of your character state evolution analysis, using the terminology you learned in class.

This section should also include at least one figure.
This will likely be your phylogenetic tree, but the design of the figure is up to you. 
For example, it may be a cladogram or phylogram, and may include reconstructed character states.

\paragraph{Works Cited}
Be sure to cite all programs used in your analysis. 
You should also cite the original Caminalcules or Dendrogrammaceae papers.
Additional citations should be added as appropriate. 
Any citation style will be fine as long as it is consistent throughout the paper.


\section{Grading Rubric}

10\% : Explanation of methods\\
10\% : Explanation of an appropriate comparison\\
5\% : Clarity of writing\\
30\% : Use of ``tree thinking'' language\\
20\% : Accurate interpretation of phylogeny\\
20\% : Accurate interpretation of character evolution\\
5\% : Works cited and in-text citations\\



\end{document}